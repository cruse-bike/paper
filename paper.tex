% Options for packages loaded elsewhere
\PassOptionsToPackage{unicode}{hyperref}
\PassOptionsToPackage{hyphens}{url}
\PassOptionsToPackage{dvipsnames,svgnames,x11names}{xcolor}
%
\documentclass[
  super,
  preprint,
  3p]{elsarticle}

\usepackage{amsmath,amssymb}
\usepackage{iftex}
\ifPDFTeX
  \usepackage[T1]{fontenc}
  \usepackage[utf8]{inputenc}
  \usepackage{textcomp} % provide euro and other symbols
\else % if luatex or xetex
  \usepackage{unicode-math}
  \defaultfontfeatures{Scale=MatchLowercase}
  \defaultfontfeatures[\rmfamily]{Ligatures=TeX,Scale=1}
\fi
\usepackage{lmodern}
\ifPDFTeX\else  
    % xetex/luatex font selection
\fi
% Use upquote if available, for straight quotes in verbatim environments
\IfFileExists{upquote.sty}{\usepackage{upquote}}{}
\IfFileExists{microtype.sty}{% use microtype if available
  \usepackage[]{microtype}
  \UseMicrotypeSet[protrusion]{basicmath} % disable protrusion for tt fonts
}{}
\makeatletter
\@ifundefined{KOMAClassName}{% if non-KOMA class
  \IfFileExists{parskip.sty}{%
    \usepackage{parskip}
  }{% else
    \setlength{\parindent}{0pt}
    \setlength{\parskip}{6pt plus 2pt minus 1pt}}
}{% if KOMA class
  \KOMAoptions{parskip=half}}
\makeatother
\usepackage{xcolor}
\setlength{\emergencystretch}{3em} % prevent overfull lines
\setcounter{secnumdepth}{5}
% Make \paragraph and \subparagraph free-standing
\ifx\paragraph\undefined\else
  \let\oldparagraph\paragraph
  \renewcommand{\paragraph}[1]{\oldparagraph{#1}\mbox{}}
\fi
\ifx\subparagraph\undefined\else
  \let\oldsubparagraph\subparagraph
  \renewcommand{\subparagraph}[1]{\oldsubparagraph{#1}\mbox{}}
\fi


\providecommand{\tightlist}{%
  \setlength{\itemsep}{0pt}\setlength{\parskip}{0pt}}\usepackage{longtable,booktabs,array}
\usepackage{calc} % for calculating minipage widths
% Correct order of tables after \paragraph or \subparagraph
\usepackage{etoolbox}
\makeatletter
\patchcmd\longtable{\par}{\if@noskipsec\mbox{}\fi\par}{}{}
\makeatother
% Allow footnotes in longtable head/foot
\IfFileExists{footnotehyper.sty}{\usepackage{footnotehyper}}{\usepackage{footnote}}
\makesavenoteenv{longtable}
\usepackage{graphicx}
\makeatletter
\def\maxwidth{\ifdim\Gin@nat@width>\linewidth\linewidth\else\Gin@nat@width\fi}
\def\maxheight{\ifdim\Gin@nat@height>\textheight\textheight\else\Gin@nat@height\fi}
\makeatother
% Scale images if necessary, so that they will not overflow the page
% margins by default, and it is still possible to overwrite the defaults
% using explicit options in \includegraphics[width, height, ...]{}
\setkeys{Gin}{width=\maxwidth,height=\maxheight,keepaspectratio}
% Set default figure placement to htbp
\makeatletter
\def\fps@figure{htbp}
\makeatother

\makeatletter
\makeatother
\makeatletter
\makeatother
\makeatletter
\@ifpackageloaded{caption}{}{\usepackage{caption}}
\AtBeginDocument{%
\ifdefined\contentsname
  \renewcommand*\contentsname{Table of contents}
\else
  \newcommand\contentsname{Table of contents}
\fi
\ifdefined\listfigurename
  \renewcommand*\listfigurename{List of Figures}
\else
  \newcommand\listfigurename{List of Figures}
\fi
\ifdefined\listtablename
  \renewcommand*\listtablename{List of Tables}
\else
  \newcommand\listtablename{List of Tables}
\fi
\ifdefined\figurename
  \renewcommand*\figurename{Figure}
\else
  \newcommand\figurename{Figure}
\fi
\ifdefined\tablename
  \renewcommand*\tablename{Table}
\else
  \newcommand\tablename{Table}
\fi
}
\@ifpackageloaded{float}{}{\usepackage{float}}
\floatstyle{ruled}
\@ifundefined{c@chapter}{\newfloat{codelisting}{h}{lop}}{\newfloat{codelisting}{h}{lop}[chapter]}
\floatname{codelisting}{Listing}
\newcommand*\listoflistings{\listof{codelisting}{List of Listings}}
\makeatother
\makeatletter
\@ifpackageloaded{caption}{}{\usepackage{caption}}
\@ifpackageloaded{subcaption}{}{\usepackage{subcaption}}
\makeatother
\makeatletter
\@ifpackageloaded{tcolorbox}{}{\usepackage[skins,breakable]{tcolorbox}}
\makeatother
\makeatletter
\@ifundefined{shadecolor}{\definecolor{shadecolor}{rgb}{.97, .97, .97}}
\makeatother
\makeatletter
\makeatother
\makeatletter
\makeatother
\journal{Journal Name}
\ifLuaTeX
  \usepackage{selnolig}  % disable illegal ligatures
\fi
\usepackage[]{natbib}
\bibliographystyle{elsarticle-num}
\IfFileExists{bookmark.sty}{\usepackage{bookmark}}{\usepackage{hyperref}}
\IfFileExists{xurl.sty}{\usepackage{xurl}}{} % add URL line breaks if available
\urlstyle{same} % disable monospaced font for URLs
\hypersetup{
  pdftitle={CRUSE to Safe Cycling in Ireland},
  pdfauthor={Dr.~Robin Lovelace; Dr.~Joey Talbot; Eugeni Vidal Tortosa; Hussein Mahfouz; Dan Brennan; Dr.~Suzanne Meade; Elaine Brick; Peter Wright},
  pdfkeywords={Cycling, Open-source, Road Safety, Active Travel},
  colorlinks=true,
  linkcolor={blue},
  filecolor={Maroon},
  citecolor={Blue},
  urlcolor={Blue},
  pdfcreator={LaTeX via pandoc}}

\setlength{\parindent}{6pt}
\begin{document}

\begin{frontmatter}
\title{CRUSE to Safe Cycling in Ireland \\\large{An Open Source
Methodology to Support Active Travel} }
\author[1]{Dr.~Robin Lovelace%
\corref{cor1}%
}
 \ead{r.lovelace@leeds.ac.uk} 
\author[1]{Dr.~Joey Talbot%
%
}

\author[1]{Eugeni Vidal Tortosa%
%
}

\author[1]{Hussein Mahfouz%
%
}

\author[2]{Dan Brennan%
%
}

\author[2]{Dr.~Suzanne Meade%
%
}

\author[3]{Elaine Brick%
%
}

\author[4]{Peter Wright%
%
}


\affiliation[1]{organization={University of
Leeds},addressline={Institute for Transport Studies, Leeds, LS2 9JT,
United Kingdom},postcodesep={}}
\affiliation[2]{organization={Transport Infrastructure
Ireland},addressline={Parkgate Business Centre, Parkgate Street, Dublin
8, D08 DK10, Ireland},postcodesep={}}
\affiliation[3]{organization={AECOM},addressline={Unit 6, Galway
Technology Park, Parkmore, Galway, Ireland},postcodesep={}}
\affiliation[4]{organization={AECOM},addressline={Winslade House,
Winslade Park, Manor Drive, Clyst St Mary, EXETER, EX5 1FY,
UK},postcodesep={}}

\cortext[cor1]{Corresponding author}








        
\begin{abstract}
Under the EU Road infrastructure safety management (RISM) directive, the
National Road Safety Strategy (RSS), and the Climate Action Plan
Transport Infrastructure Ireland (TII) has a remit for road safety and
decarbonizing a predominantly road-based network in Ireland.

To address data needs for both safety and project evaluation on the
National Road Network (NRN), the Cycle Route Uptake and Scenario
Estimation (CRUSE) Tool was developed. While cycling in Ireland
represents only 3\% of total modal share, with higher intensities in
urban areas, the levels of cycling collisions are disproportionately
high at 20\% of all serious injuries and 7\% of all fatalities. If
Ireland is to meet its climate and safety targets, data to establish
baseline cycling levels and future cycling levels is needed.

Due to an absence of reliable data, particularly rural cycling levels,
TII commissioned the Institute for Transport Studies (ITS) at the
University of Leeds and AECOM to develop a new tool for this purpose.
ITS Leeds led the development of the PCT for England and Wales, which
has ``revolutionized the practice of strategic cycle planning
nationally''. The tool is an open-source approach using recognized
open-source methodology to enable planners, engineers, and other
stakeholders to make evidence-based decisions for the NRN. CRUSE is
available at \url{https://cruse.bike/} and builds on the Propensity to
Cycle Tool (PCT) for England and Wales. CRUSE goes beyond the PCT in
several important ways, higher resolution data, more trip types,
including estimates for education, inter-urban, and recreational trips.
In addition to understanding cycling intensity, for asset planning and
management purposes, the tool provides essential road safety information
to enable reporting of disaggregate collision rates.

CRUSE is structured in a similar way to the traditional four-stage
transport model, but its use of Open Street Map (OSM) data, used by
\href{https://www.cyclestreets.net/}{Cyclestreets.net} for routing,
enables network quality to be assessed without costly surveys to record
new infrastructure. OSM tags generate ``cycle friendliness'' estimates
of all links on the network, based on existing recorded infrastructure.
A range of networks is provided, highlighting routes for directness
(Fastest) and ``cycle friendliness'' (Quietest). It uses origin and
destination data from the 2016 Census in combination with modeled demand
data to estimate cycling levels and potential at the area, route, and
network levels for each county in Ireland and offers estimates of the
baseline level of cycling and several future scenario-based levels of
cycling.

As countries, like Ireland, invest in cycling, the number of killed and
seriously injured cyclists must reduce too. The CRUSE Tool provides
estimates of cycling potential and routing for each county in Ireland,
and works in both urban and rural settings, to enable monitoring of
cycling safety. With growth in the E-bike market, the tool will help
inform inter-urban and rural networks to support the transfer of trips
to sustainable modes for longer journeys. The CRUSE Tool methodology and
findings are directly relevant to addressing the challenges and
opportunities faced by other NRAs. The datasets resulting from the
project are open access and can be used by both non-experts and
professionals.
\end{abstract}





\begin{keyword}
    Cycling \sep Open-source \sep Road Safety \sep 
    Active Travel
\end{keyword}
\end{frontmatter}
    \ifdefined\Shaded\renewenvironment{Shaded}{\begin{tcolorbox}[interior hidden, breakable, sharp corners, boxrule=0pt, frame hidden, borderline west={3pt}{0pt}{shadecolor}, enhanced]}{\end{tcolorbox}}\fi

\newpage{}

\hypertarget{introduction}{%
\section{Introduction}\label{introduction}}

Transport is a major contributor to climate change, premature death due
to road traffic collisions and air pollution, yet has the potential to
be a major contributor to decarbonization and improved public health.
Transport is responsible for 23\% of global emissions, 70\% of which is
from road transport, nearly half of which is from passenger cars
\citep{jaramillo2022}. The transport system encourages, enables and in
some cases enforces unsustainable lifestyles, including over-consumption
of goods due to excessive mobile storage space and dependency on
services that are only accessible by car due to land use plans that have
built up around roads \citep{gray2001, shergold2012, motte-baumvol2010}.

Recognizing the impacts of poorly designed and performing transport
systems on their citizens, governments in many countries have set
targets and taken actions and set targets. In the context of climate,
road safety and physicial inactivity crises, policies to improve
transport systems can be classified according to the
`Avoid-Shift-Improve' (ASI) framework \citep{jaramillo2022}. The
framework highlights the importance of demand reduction (\emph{avoid}ing
unnecessary trips), in addition to mode \emph{shift} and
\emph{improvement} of existing energy converters, in that order.

Uptake of cycling, the main topic of this paper, should be seen in this
broader context of transport decarbonisation. Although cycling uptake
appears on the surface to only relate to the `shift' part of the ASI
framework, closer consideration of the knock-on impacts of cycling
uptake shows that it can also help avoid unnecessary trips {[}citation
needed, any ideas Hussein?{]} . The rapid uptake of highly efficient
e-bikes can also be seen as an improvement on most electric vehicles,
which are too heavy and expensive to be a sustainable alternative and
could in fact delay the transition away from car dependency {[}citation
needed{]}.

At the European Union level \ldots{}

In Ireland\ldots{}

{[}Gary, Suzanne, others input here please.{]}

\hypertarget{methods}{%
\section{Methods}\label{methods}}

{[}Should this bit go in a subsection in the Introduction rather than in
the methods? (RL){]}

The methods used to generate the evidence presented in the CRUSE tool
for Ireland build on previous work, particularly the Propensity to Cycle
Tool (PCT), which was originally developed for the UK's Department for
Transport and which is publicly available at www.pct.bike
\citep{lovelace2017}. The PCT approach has had major policy impacts, as
outlined in Research Excellence Framework (REF) impact case studies
submitted by the University of Leeds and the University of Westminster,
which demonstrate that the tool ``revolutionised strategic cycle
planning in England and Wales''\footnote{See REF Impact Case Study
  ``Cycle network policy, planning and investment transformed by the
  Propensity to Cycle Tool'' at
  \href{https://results2021.ref.ac.uk/impact/847d1191-7f25-46ba-a399-b481125edc8f}{results2021.ref.ac.uk}
  submitted by the University of Leeds.} by overcoming the barriers to
cycling investment imposed by lack of evidence on cycling
potential\footnote{See the REF Impact Case Study ``Creating Step Changes
  in Cycling Policy and Infrastructure Planning across the UK'' at
  \href{https://results2021.ref.ac.uk/impact/4BBF3436-FD10-4C75-9791-F5E98AB4411B}{results2021.ref.ac.uk}
  submitted by the University of Westminster.} .

The first version of the PCT was based solely on current and future
potential uptake of cycling for single stage travel to work at desire
line, zone, route, and route network levels \citep{lovelace2016}. It was
launched in April 2017 as the government-endorsed tool for strategic
cycle network planning, as part of the Cycling and Walking Investment
Strategy \citep{cycling2017}. Extensions of the PCT approach have
included estimation of benefits at the individual level
\citep{woodcock2018}, addition of travel to school network
\citep{goodman2019}, and improvement modelling of impacts on health,
environmental and distributional outcomes \citep{woodcock2021}. In
Portugal, a cycling uptake tool to support decision making was developed
based on PCT methods, for the Lisbon metro region (see
\href{https://biclar.tmlmobilidade.pt}{biclaR}, and again, social and
environment impacts were added as a relevant information to support
decisions, using the WHO HEAT for Cycling tool\footnote{See the Health
  economic assessment tool for walking and cycling at
  \href{https://www.heatwalkingcycling.org/\#homepage}{www.heatwalkingcycling.org}.}
\citep{felix2023}. biclaR added on the previous methods by setting an
`intermodality' scenario that combines cycling with currently available
public transit options based on General Transit Feed Specification
(GTFS) data.

Bit on Jittering \citep{lovelace2022}

Despite these advances, the PCT approach outlined in previous research
paper had limitations:

\begin{itemize}
\item
  Limited coverage of modes beyond travel to work and school
\item
  \ldots{}
\end{itemize}

\hypertarget{results}{%
\section{Results}\label{results}}

\hypertarget{discussion}{%
\section{Discussion}\label{discussion}}

\hypertarget{conclusions}{%
\section{Conclusions}\label{conclusions}}

\hypertarget{list-of-abbreviations}{%
\section{List of abbreviations}\label{list-of-abbreviations}}

\hypertarget{declarations}{%
\section{Declarations}\label{declarations}}

\hypertarget{availability-of-data-and-material}{%
\subsection*{Availability of data and
material}\label{availability-of-data-and-material}}
\addcontentsline{toc}{subsection}{Availability of data and material}

\hypertarget{funding}{%
\subsection*{Funding}\label{funding}}
\addcontentsline{toc}{subsection}{Funding}

\hypertarget{acknowledgements}{%
\subsection*{Acknowledgements}\label{acknowledgements}}
\addcontentsline{toc}{subsection}{Acknowledgements}

\hypertarget{competing-interests}{%
\subsection*{Competing interests}\label{competing-interests}}
\addcontentsline{toc}{subsection}{Competing interests}

\hypertarget{authors-contributions}{%
\subsection*{Authors' contributions}\label{authors-contributions}}
\addcontentsline{toc}{subsection}{Authors' contributions}


  \bibliography{bibliography.bib}


\end{document}
