% Options for packages loaded elsewhere
\PassOptionsToPackage{unicode}{hyperref}
\PassOptionsToPackage{hyphens}{url}
\PassOptionsToPackage{dvipsnames,svgnames,x11names}{xcolor}
%
\documentclass[
  super,
  preprint,
  3p]{elsarticle}

\usepackage{amsmath,amssymb}
\usepackage{iftex}
\ifPDFTeX
  \usepackage[T1]{fontenc}
  \usepackage[utf8]{inputenc}
  \usepackage{textcomp} % provide euro and other symbols
\else % if luatex or xetex
  \usepackage{unicode-math}
  \defaultfontfeatures{Scale=MatchLowercase}
  \defaultfontfeatures[\rmfamily]{Ligatures=TeX,Scale=1}
\fi
\usepackage{lmodern}
\ifPDFTeX\else  
    % xetex/luatex font selection
\fi
% Use upquote if available, for straight quotes in verbatim environments
\IfFileExists{upquote.sty}{\usepackage{upquote}}{}
\IfFileExists{microtype.sty}{% use microtype if available
  \usepackage[]{microtype}
  \UseMicrotypeSet[protrusion]{basicmath} % disable protrusion for tt fonts
}{}
\makeatletter
\@ifundefined{KOMAClassName}{% if non-KOMA class
  \IfFileExists{parskip.sty}{%
    \usepackage{parskip}
  }{% else
    \setlength{\parindent}{0pt}
    \setlength{\parskip}{6pt plus 2pt minus 1pt}}
}{% if KOMA class
  \KOMAoptions{parskip=half}}
\makeatother
\usepackage{xcolor}
\setlength{\emergencystretch}{3em} % prevent overfull lines
\setcounter{secnumdepth}{5}
% Make \paragraph and \subparagraph free-standing
\ifx\paragraph\undefined\else
  \let\oldparagraph\paragraph
  \renewcommand{\paragraph}[1]{\oldparagraph{#1}\mbox{}}
\fi
\ifx\subparagraph\undefined\else
  \let\oldsubparagraph\subparagraph
  \renewcommand{\subparagraph}[1]{\oldsubparagraph{#1}\mbox{}}
\fi


\providecommand{\tightlist}{%
  \setlength{\itemsep}{0pt}\setlength{\parskip}{0pt}}\usepackage{longtable,booktabs,array}
\usepackage{calc} % for calculating minipage widths
% Correct order of tables after \paragraph or \subparagraph
\usepackage{etoolbox}
\makeatletter
\patchcmd\longtable{\par}{\if@noskipsec\mbox{}\fi\par}{}{}
\makeatother
% Allow footnotes in longtable head/foot
\IfFileExists{footnotehyper.sty}{\usepackage{footnotehyper}}{\usepackage{footnote}}
\makesavenoteenv{longtable}
\usepackage{graphicx}
\makeatletter
\def\maxwidth{\ifdim\Gin@nat@width>\linewidth\linewidth\else\Gin@nat@width\fi}
\def\maxheight{\ifdim\Gin@nat@height>\textheight\textheight\else\Gin@nat@height\fi}
\makeatother
% Scale images if necessary, so that they will not overflow the page
% margins by default, and it is still possible to overwrite the defaults
% using explicit options in \includegraphics[width, height, ...]{}
\setkeys{Gin}{width=\maxwidth,height=\maxheight,keepaspectratio}
% Set default figure placement to htbp
\makeatletter
\def\fps@figure{htbp}
\makeatother

\makeatletter
\makeatother
\makeatletter
\makeatother
\makeatletter
\@ifpackageloaded{caption}{}{\usepackage{caption}}
\AtBeginDocument{%
\ifdefined\contentsname
  \renewcommand*\contentsname{Table of contents}
\else
  \newcommand\contentsname{Table of contents}
\fi
\ifdefined\listfigurename
  \renewcommand*\listfigurename{List of Figures}
\else
  \newcommand\listfigurename{List of Figures}
\fi
\ifdefined\listtablename
  \renewcommand*\listtablename{List of Tables}
\else
  \newcommand\listtablename{List of Tables}
\fi
\ifdefined\figurename
  \renewcommand*\figurename{Figure}
\else
  \newcommand\figurename{Figure}
\fi
\ifdefined\tablename
  \renewcommand*\tablename{Table}
\else
  \newcommand\tablename{Table}
\fi
}
\@ifpackageloaded{float}{}{\usepackage{float}}
\floatstyle{ruled}
\@ifundefined{c@chapter}{\newfloat{codelisting}{h}{lop}}{\newfloat{codelisting}{h}{lop}[chapter]}
\floatname{codelisting}{Listing}
\newcommand*\listoflistings{\listof{codelisting}{List of Listings}}
\makeatother
\makeatletter
\@ifpackageloaded{caption}{}{\usepackage{caption}}
\@ifpackageloaded{subcaption}{}{\usepackage{subcaption}}
\makeatother
\makeatletter
\@ifpackageloaded{tcolorbox}{}{\usepackage[skins,breakable]{tcolorbox}}
\makeatother
\makeatletter
\@ifundefined{shadecolor}{\definecolor{shadecolor}{rgb}{.97, .97, .97}}
\makeatother
\makeatletter
\makeatother
\makeatletter
\makeatother
\journal{Journal Name}
\ifLuaTeX
  \usepackage{selnolig}  % disable illegal ligatures
\fi
\usepackage[]{natbib}
\bibliographystyle{elsarticle-num}
\IfFileExists{bookmark.sty}{\usepackage{bookmark}}{\usepackage{hyperref}}
\IfFileExists{xurl.sty}{\usepackage{xurl}}{} % add URL line breaks if available
\urlstyle{same} % disable monospaced font for URLs
\hypersetup{
  pdftitle={CRUSE to Safe Cycling in Ireland},
  pdfauthor={Dr.~Robin Lovelace; Dr.~Joey Talbot; Eugeni Vidal Tortosa; Hussein Mahfouz; Dan Brennan; Dr.~Suzanne Meade; Elaine Brick; Peter Wright},
  pdfkeywords={Cycling, Open-source, Road Safety, Active Travel},
  colorlinks=true,
  linkcolor={blue},
  filecolor={Maroon},
  citecolor={Blue},
  urlcolor={Blue},
  pdfcreator={LaTeX via pandoc}}

\setlength{\parindent}{6pt}
\begin{document}

\begin{frontmatter}
\title{CRUSE to Safe Cycling in Ireland \\\large{An Open Source
Methodology to Support Active Travel} }
\author[1]{Dr.~Robin Lovelace%
\corref{cor1}%
}
 \ead{r.lovelace@leeds.ac.uk} 
\author[1]{Dr.~Joey Talbot%
%
}

\author[1]{Eugeni Vidal Tortosa%
%
}

\author[1]{Hussein Mahfouz%
%
}

\author[2]{Dan Brennan%
%
}

\author[2]{Dr.~Suzanne Meade%
%
}

\author[3]{Elaine Brick%
%
}

\author[4]{Peter Wright%
%
}


\affiliation[1]{organization={University of
Leeds},addressline={Institute for Transport Studies, Leeds, LS2 9JT,
United Kingdom},postcodesep={}}
\affiliation[2]{organization={Transport Infrastructure
Ireland},addressline={Parkgate Business Centre, Parkgate Street, Dublin
8, D08 DK10, Ireland},postcodesep={}}
\affiliation[3]{organization={AECOM},addressline={Unit 6, Galway
Technology Park, Parkmore, Galway, Ireland},postcodesep={}}
\affiliation[4]{organization={AECOM},addressline={Winslade House,
Winslade Park, Manor Drive, Clyst St Mary, EXETER, EX5 1FY,
UK},postcodesep={}}

\cortext[cor1]{Corresponding author}








        
\begin{abstract}
Under the EU Road infrastructure safety management (RISM) directive, the
National Road Safety Strategy (RSS), and the Climate Action Plan
Transport Infrastructure Ireland (TII) has a remit for road safety and
decarbonizing a predominantly road-based network in Ireland.

To address data needs for both safety and project evaluation on the
National Road Network (NRN), the Cycle Route Uptake and Scenario
Estimation (CRUSE) Tool was developed. While cycling in Ireland
represents only 3\% of total modal share, with higher intensities in
urban areas, the levels of cycling collisions are disproportionately
high at 20\% of all serious injuries and 7\% of all fatalities. If
Ireland is to meet its climate and safety targets, data to establish
baseline cycling levels and future cycling levels is needed.

Due to an absence of reliable data, particularly rural cycling levels,
TII commissioned the Institute for Transport Studies (ITS) at the
University of Leeds and AECOM to develop a new tool for this purpose.
ITS Leeds led the development of the PCT for England and Wales, which
has ``revolutionized the practice of strategic cycle planning
nationally''. The tool is an open-source approach using recognized
open-source methodology to enable planners, engineers, and other
stakeholders to make evidence-based decisions for the NRN. CRUSE is
available at \url{https://cruse.bike/} and builds on the Propensity to
Cycle Tool (PCT) for England and Wales. CRUSE goes beyond the PCT in
several important ways, higher resolution data, more trip types,
including estimates for education, inter-urban, and recreational trips.
In addition to understanding cycling intensity, for asset planning and
management purposes, the tool provides essential road safety information
to enable reporting of disaggregate collision rates.

CRUSE is structured in a similar way to the traditional four-stage
transport model, but its use of Open Street Map (OSM) data, used by
\href{https://www.cyclestreets.net/}{Cyclestreets.net} for routing,
enables network quality to be assessed without costly surveys to record
new infrastructure. OSM tags generate ``cycle friendliness'' estimates
of all links on the network, based on existing recorded infrastructure.
A range of networks is provided, highlighting routes for directness
(Fastest) and ``cycle friendliness'' (Quietest). It uses origin and
destination data from the 2016 Census in combination with modeled demand
data to estimate cycling levels and potential at the area, route, and
network levels for each county in Ireland and offers estimates of the
baseline level of cycling and several future scenario-based levels of
cycling.

As countries, like Ireland, invest in cycling, the number of killed and
seriously injured cyclists must reduce too. The CRUSE Tool provides
estimates of cycling potential and routing for each county in Ireland,
and works in both urban and rural settings, to enable monitoring of
cycling safety. With growth in the E-bike market, the tool will help
inform inter-urban and rural networks to support the transfer of trips
to sustainable modes for longer journeys. The CRUSE Tool methodology and
findings are directly relevant to addressing the challenges and
opportunities faced by other NRAs. The datasets resulting from the
project are open access and can be used by both non-experts and
professionals.
\end{abstract}





\begin{keyword}
    Cycling \sep Open-source \sep Road Safety \sep 
    Active Travel
\end{keyword}
\end{frontmatter}
    \ifdefined\Shaded\renewenvironment{Shaded}{\begin{tcolorbox}[enhanced, frame hidden, interior hidden, borderline west={3pt}{0pt}{shadecolor}, sharp corners, boxrule=0pt, breakable]}{\end{tcolorbox}}\fi

\hypertarget{th-transport-research-arena-tra-2024-dublin-ireland}{%
\subsection*{10th Transport Research Arena, TRA 2024, Dublin,
Ireland}\label{th-transport-research-arena-tra-2024-dublin-ireland}}
\addcontentsline{toc}{subsection}{10th Transport Research Arena, TRA
2024, Dublin, Ireland}

\newpage{}

\hypertarget{introduction}{%
\section{Introduction}\label{introduction}}

\citep{lovelace2017}

• Title page/main information (title, full names and institutional
addresses for all authors, indication of the corresponding author) •
Abstract • Introduction • Results • Discussion (can be combined in
`Results and Discussion' section) • Conclusions • Methods/Experimental
(can also be placed after Introduction) • List of abbreviations •
Declarations\emph{ o Availability of data and material o Funding o
Acknowledgements o Competing interests o Authors' contributions o
Authors' information (optional) o }If any of the sections are not relev


  \bibliography{bibliography.bib}


\end{document}
